%\special{pdf: BOP q 0.6 0.8 1.0 rg 0 0 m 612 0 l 612 792 l 0 792 l b Q  }
%\special{pdf: BOP q 0.6 0.8 1.0 rg 36 36 m 576 36 l 576 756 l 36 756 l b Q  }
% \special{  pdf:  DOCINFO <</Producer (Me) /Author (Mark A. Wicks) /Title (Some sort of title.)>>}%
%\special{  pdf:  DOCVIEW <</PageMode /UseThumbs>>}%
\special{  pdf:  DOCVIEW <</PageMode /UseOutlines>>}%
\catcode`\ =11\def\space{ }\catcode`\ =10
\def\colored#1#2{%
   \special{pdf:CONTENT q #1 rg }%
   #2\special{pdf:CONTENT Q }}%
\def\red#1{\colored{1 0 0}{#1}}%
\def\green#1{\colored{0 1 0}{#1}}%
\def\blue#1{\colored{0 0 1}{#1}}%
\def\yellow#1{\colored{0.8 0.9 0.1}{#1}}%
\font\maintitlefont=cmr12 at 17.28pt
\font\headingfont=cmr12 at 14.4pt
% Variables declared here
% Miscellaneous token lists
\newtoks\title\newtoks\author
% Section levels
\newcount\sectioncount\newcount\ssectioncount\newcount\sssectioncount
\sectioncount0\ssectioncount0\sssectioncount0
\def\settitle{%
  \centerline{\maintitlefont\blue{\the\title}}%
  \special {pdf:  DOCINFO << /Title (\the\title) >>}}%
\def\setauthor{%
  \centerline{\headingfont\blue{\the\author}}%
  \special {pdf:  DOCINFO << /Author (\the\author) >>}}%
\def\setheading#1{%
  {\headingfont\blue{#1}}\raise\baselineskip\hbox{\special{pdf: OUTLINE 1 << /Title (#1) /Dest [
@thispage /FitH @ypos ]  >> }}}%
\def\setsubheading#1{%
  {\headingfont\blue{#1}}\raise\baselineskip\hbox{\special{pdf: OUTLINE 2 << /Title (#1) /Dest [
@thispage /FitH @ypos ]  >> }}}%
\def\maketitle{\settitle\bigskip\setauthor\bigskip}
\def\section#1{\advance\sectioncount by 1\ssectioncount0%
\bigskip\noindent\setheading{\the\sectioncount. #1}\par\nobreak\medskip}%
\def\subsection#1{\advance\ssectioncount by 1%
\bigskip\noindent\setsubheading{\the\sectioncount.\the\ssectioncount\ #1}\par\nobreak\medskip}%
\def\dvipdf{{\tt dvipdf}}%
\def\version{0.5}%
\title{The dvipdf, version \version\ User's Manual}
\author{Mark A. Wicks}
\maketitle
\section{Background}
I became interested in DVI to PDF conversion in October, 1998.  At that
time, I examined the possible alternatives and investigated what other
people were doing. I concluded that the most widely accepted method was to
use Adobe's Acrobat distiller on a Postscript file produced by {\tt dvips}.
The hyperlink features are accessed by using \TeX\ $\tt \backslash${\tt
special}s to embed pdfmarks in the Postscript produced by {\tt dvips}.
Han The Than's PDF\TeX project is an alternative solution.
Although quite good and fairly mature, the PDF\TeX project required
modifying the \TeX\ source code to add primitives to support the PDF features.
I have a firm belief that \TeX should remain pristine
unless a compelling case can be made that certain features cannot
be implemented with \TeX\ $\tt \backslash${\tt special}s.  At least
one other DVI to PDF project exists, but it wasn't widely available.

From a technical standpoint,
I believe that using distiller is and will be the best
solution for some time.  However I had several objections to the use of distiller.
My principle objection is that it isn't available for Linux---my principle
operating system.  Also, the conversion to Postscript as an intermediate
step seems unnatural.  \TeX is a complete programming language.
The DVI specification is a page description language.
Postscript is a programming language, while PDF is a page description language.
Why convert a page description into a program for generating
that page only so you can run a distiller that generates
the page description obtained by running that program?
I would argue that \TeX\ is a analogous to postscript (without the graphics)
while DVI is analogous to PDF (without graphics or hyperlinks).
Pdfmarks are postscript features meant for the distiller that are roughly analogous
to \TeX $\tt \backslash${\tt special}s, which are meant for the DVI driver.
It seems natural to have a DVI driver with features similar to those of the
distiller.  My goal in writing \dvipdf\ was to convert a DVI file directly into PDF, while
preserving the flavor of the pdfmark
interface, so that a user familiar with the pdfmark interface
will have little trouble adapting to \dvipdf.

Unfortunately the ability to include graphics
is and will likely remain a sensible reason to continue to
use postscript only because so many graphics programs will produce an encapsulated postscript
file.  I would hope that in the future, graphics programs
will produce PDF content streams, or PDF objects that
may be included into a DVI to PDF translator.

\section{Introduction}
This document describes and serves as an example input file for \dvipdf~version~\version.
It assumes some familiarity with PDF.

\section{Functions analogous to PDFmarks}
These functions are all executed via \TeX\ $\tt \backslash${\tt special}s
prefixed with {\tt pdf:}, e.g.,

{\tt pdf: OUT 1 << /Title (Introduction) /Dest [ 1 0 R /FitH 234 ] >> }

\subsection{ANN}
\subsection{OUT}
OUT {\it level} {\it dictionary}

The parameter {\it level} is an integer representing the level of the outline
entry (beginning with 1) and {\it dictionary} must contain
the two keys {\tt /Title} and either {\tt /Dest} or {\tt /A}.
It may also contain the {\tt /AA} key.  These keys are documented
in the PDF Reference Manual.

%\subsection{ARTICLE}
%Currently, the ARTICLE command is not implemented.  It would 
%facilitate the construction of threads and beads.

%\subsection{DEST}
%Currently, named destinations are not supported.

\subsection{DOCINFO}

DOCINFO {\it dictionary}

DOCINFO adds the keys in the specified dictionary to the
document's Info dictionary.  All keys are optional, but may include
the keys {\tt Author}, {\tt Title}, {\tt Keywords}, {\tt Subject},
and {\tt Creator}.

\subsection{DOCVIEW}
DOCVIEW {\it dictionary}

DOCVIEW adds the keys in the specified dictionary to the
document's Catalog dictionary.  All keys are optional, but may include
the keys {\tt /PageMode},
{\tt /URI}, {\tt /OpenAction}, {\tt /AA}
and {\tt ViewerPreferences}.  See the PDF Reference Manual
for documentation of these keys and additional keys.

\subsection{OBJ}

OBJ [@{\it name}] {\it object}

OBJ creates a PDF object.  The parameter {\it object} is any valid PDF object.  The parameter @{\it name}
may be used to refer to this object within other objects.  It
will be expanded in any {\tt special} where a PDF object is expected.
Typically {\it object} is an array or dictionary.  It may be an empty array or
dictionary that may be accumulated dynamically via the PUT command.

\subsection{PUT}
PUT @{\it name} {\it object}

or 

PUT @{\it name} {\it dictionary}

PUT modifies an existing PDF object created with OBJ.
The first form is used when @{\tt name} is an array.  The second
form is used when @{\tt name} is a dictionary.  Arrays are
incremented one object at a time.  All keys in {\it dictionary}
are added to the dictionary represented by @{\it name}.

\subsection{CLOSE}

CLOSE @{\it name}

CLOSE writes a PDF object created with OBJ to the PDF file.
No further PUT commands may be executed for this object.
The object may continue to be referenced using @{\it name}
indefinitely.

\section{Additional functions}
\subsection{BOP}

BOP {\tt stream}

BOP specifies a marking stream to be generated at the top of each page.


\subsection{EOP}

BOP {\tt stream}

EOP specifies a marking stream to be generated at the top of each page.

\bye
