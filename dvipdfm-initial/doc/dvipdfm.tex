%\special{pdf: bop q 0.6 0.8 1.0 rg 0 0 m 612 0 l 612 792 l 0 792 l b Q  }
%\special{pdf: bop q 0.6 0.8 1.0 rg 36 36 m 576 36 l 576 756 l 36 756 l b Q  }
% \special{  pdf:  docinfo <</Producer (Me) /Author (Mark A. Wicks) /Title (Some sort of title.)>>}%
%\special{  pdf:  docview <</PageMode /UseThumbs>>}%
\special{  pdf:  docview <</PageMode /UseOutlines>>}%
\catcode`\ =11\def\space{ }\catcode`\ =10
\def\colored#1#2{%
   \special{pdf:content q #1 rg }%
   #2\special{pdf:content Q }}%
\def\red#1{\colored{1 0 0}{#1}}%
\def\green#1{\colored{0 1 0}{#1}}%
\def\blue#1{\colored{0 0 1}{#1}}%
\def\yellow#1{\colored{0.8 0.9 0.1}{#1}}%
\font\maintitlefont=cmr12 at 17.28pt
\font\headingfont=cmr12 at 14.4pt
% Variables declared here
% Miscellaneous token lists
\newtoks\title\newtoks\author
% Section levels
\newcount\sectioncount\newcount\ssectioncount\newcount\sssectioncount
\sectioncount0\ssectioncount0\sssectioncount0
\def\settitle{%
  \centerline{\maintitlefont\blue{\the\title}}%
  \special {pdf:  docinfo << /Title (\expandafter\the\title) >>}}%
\def\setauthor{%
  \centerline{\headingfont\blue{\the\author}}%
  \special {pdf:  docinfo << /Author (\the\author) >>}}%
\def\setheading#1{%
  {\headingfont\blue{#1}}\raise\baselineskip\hbox{\special{pdf: outline 1 << /Title (#1) /Dest [
@thispage /FitH @ypos ]  >> }}}%
\def\setsubheading#1{%
  {\headingfont\blue{#1}}\raise\baselineskip\hbox{\special{pdf: outline 2 << /Title (#1) /Dest [
@thispage /FitH @ypos ]  >> }}}%
\def\maketitle{\settitle\bigskip\setauthor\bigskip}
\def\section#1{\advance\sectioncount by 1\ssectioncount0%
\bigskip\noindent\setheading{\the\sectioncount. #1}\par\nobreak\medskip}%
\def\subsection#1{\advance\ssectioncount by 1%
\bigskip\noindent\setsubheading{\the\sectioncount.\the\ssectioncount\ #1}\par\nobreak\medskip}%
\def\dvipdf{{\tt dvipdfm}}%
\def\version{0.7dev}
\title{dvipdfm, version 0.7dev User's Manual}
\author{Mark A. Wicks}
\maketitle
\section{Background}
At the time I wrote {\tt dvipdfm},
the most widely accepted method to generate PDF file from \TeX\
was to use Adobe's Acrobat distiller on a Postscript file produced by {\tt dvips}.
The hyperlink features are accessed by using \TeX\ $\tt \backslash${\tt
special}s to embed pdfmarks in the Postscript produced by {\tt dvips}.
Han The Than's PDF\TeX project is an alternative solution.
Although quite good and fairly mature, the PDF\TeX project required
modifying the \TeX\ source code to add primitives to support the PDF features.
I have a firm belief that \TeX should remain pristine
unless a compelling case can be made that certain features cannot
be implemented with \TeX\ $\tt \backslash${\tt special}s.  At least
one other DVI to PDF project exists, but it wasn't widely available.

From a technical standpoint,
distiller will probably remain the
best approach for some time.
However, I have several objections to the use of distiller,
and feel people need other options.
One objection is that it isn't available for Linux---my principle
operating system.  Also, the conversion to Postscript as an intermediate
step seems unnatural.  \TeX\ is a programming language.

My second objection is philosophical.
The DVI specifies a page description language.
A DVI file contains no branching or decision instructions.
Similarly Postscript is a programming language, while PDF is a page description language
without any branching or decision capabilities.
In some sense \TeX\ is analogous to postscript (without the graphics)
while DVI is analogous to PDF (without the graphics or the hyperlinks).
Using Acrobat Distiller
requires going from page description to program back to page description.
Pdfmarks are postscript features, which are meant for the distiller, are
analogous to \TeX $\tt \backslash${\tt special}s, which are meant for the DVI driver.
It seems natural to go directly from DVI to PDF, where \TeX\ replaces
postscript and where the DVI driver replaces and implements
$\tt \backslash${\tt special}s similar to the {\tt pdfmarks} in Adobe's Acrobat Distiller.

Unfortunately, until graphics software
begins to produce PDF content streams or encapsulated
PDF objects, Postscript will remain the easiest
way to include graphics in \TeX\ documents.
I would hope that in the future, graphics programs
will produce PDF content streams, or PDF objects that
may be included into a DVI to PDF translator.  Either
of these may be easily included using {\tt dvipdfm}
or a similar driver.

\section{Introduction}
This document describes and serves as an example input file for \dvipdf~version~\version.
It assumes some familiarity with PDF.

\section{Functions analogous to PDFmarks}
These functions are all executed via \TeX\ $\tt \backslash${\tt special}s
prefixed with {\tt pdf:}, e.g.,

{\tt pdf: out 1 << /Title (Introduction) /Dest [ 1 0 R /FitH 234 ] >> }

\subsection{ann}
\subsection{out}
{\tt out} {\it level} {\it dictionary}

The parameter {\it level} is an integer representing the level of the outline
entry (beginning with 1) and {\it dictionary} must contain
the two keys {\tt /Title} and either {\tt /Dest} or {\tt /A}.
It may also contain the {\tt /AA} key.  These keys are documented
in the PDF Reference Manual.

%\subsection{ARTICLE}
%Currently, the ARTICLE command is not implemented.  It would 
%facilitate the construction of threads and beads.

%\subsection{DEST}
%Currently, named destinations are not supported.

\subsection{docinfo}

{\tt docinfo} {\it dictionary}

The {\tt docinfo} command adds the keys in the specified dictionary to the
document's Info dictionary.  All keys are optional, but may include
the keys {\tt Author}, {\tt Title}, {\tt Keywords}, {\tt Subject},
and {\tt Creator}.

\subsection{docview}
{\tt docview} {\it dictionary}

The {\tt docview} command adds the keys in the specified dictionary to the
document's Catalog dictionary.  All keys are optional, but may include
the keys {\tt /PageMode},
{\tt /URI}, {\tt /OpenAction}, {\tt /AA}
and {\tt ViewerPreferences}.  See the PDF Reference Manual
for documentation of these keys and additional keys.

\subsection{epdf}

{\tt epdf} [@{\it name}] {\it filename}

The {\tt epdf} command ``encapsulates'' the first page of a PDF
file into a PDF XObject.  The resulting XObject is drawn
at the current location of the page.  The current point
represents the lower left-hand corner of the XObject's coordinate
system.  The optional @{\tt name} parameter may be used
to reference this object with other objects.
It will be expanded to a reference for this object
within any {\tt special} where a PDF object is expected.

\subsection{obj}

{\tt obj} [@{\it name}] {\it object}

The {\tt obj} command creates a
PDF object.  The parameter {\it object} is any valid PDF object.  The parameter @{\it name}
may be used to refer to this object within other objects.
It will be expanded within any {\tt special} where a PDF object is expected.
Typically {\it object} is an array or dictionary.  It may be an empty array or
dictionary that can be constructed dynamically via the {\tt put} command.

\subsection{put}
{\tt put} @{\it name} {\it object}

or 

{\tt put} @{\it name} {\it dictionary}

The {\tt put} command modifies an existing PDF object created with OBJ.
The first form is used when @{\tt name} is an array.  The second
form is used when @{\tt name} is a dictionary.  Arrays are
incremented one object at a time.  All keys in {\it dictionary}
are added to the dictionary represented by @{\it name}.

\subsection{close}

{\tt close} @{\it name}

The {\tt close} writes a PDF object created with OBJ to the PDF file.
No further PUT commands may be executed for this object.
The object may continue to be referenced using @{\it name}
indefinitely.

\section{Additional functions}
\subsection{bop}

{\tt bop} {\tt stream}

The {\tt bop} command specifies a marking stream to be generated at the top of each page.

\subsection{eop}

{\tt eop} {\tt stream}

The {\tt eop} specifies a marking stream to be generated at the top of each page.

\bye
