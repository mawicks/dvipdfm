%\special{pdf: bop q 0.6 0.8 1.0 rg 0 0 m 612 0 l 612 792 l 0 792 l b Q  }
%\special{pdf: bop q 0.6 0.8 1.0 rg 36 36 m 576 36 l 576 756 l 36 756 l b Q  }
% \special{  pdf:  docinfo <</Producer (Me) /Author (Mark A. Wicks) /Title (Some sort of title.)>>}%
%\special{  pdf:  docview <</PageMode /UseThumbs>>}%
\special{  pdf:  docview <</PageMode /UseOutlines>>}%
\catcode`\ =11\def\space{ }\catcode`\ =10
% Page layout
\advance\hoffset by 0.5truein
\advance\voffset by 0.5truein
\advance\hsize by -1.0in
\advance\vsize by -1.0in
%
% Some color definitions
% 
\def\colored#1#2{%
  \special{pdf:bc #1}#2\special{pdf:ec}}%
\def\red#1{\colored{[1 0 0]}{#1}}%
\def\green#1{\colored{[0 1 0]}{#1}}%
\def\blue#1{\colored{[0 0 1]}{#1}}%
\def\yellow#1{\colored{[0.8 0.9 0.1]}{#1}}%
%
% Some font definitions
%
\font\maintitlefont=cmss12 at 17.28pt
\font\headingfont=cmss12 at 14.4pt
\font\versionfont=cmss12
%
% Miscellaneous token lists
%
\newtoks\title\newtoks\author\newtoks\version
\newtoks\date
%
% Counters for section levels
%
\newcount\sectioncount\newcount\ssectioncount\newcount\sssectioncount
\sectioncount0\ssectioncount0\sssectioncount0
%
\def\settitle{%
  {\maintitlefont\blue{\the\title}}%
  \special {pdf:  docinfo << /Title (\expandafter\the\title) >>}}%
%
\def\setauthor{%
  {\headingfont\blue{\the\author}}%
  \special {pdf:  docinfo << /Author (\the\author) >>}}%
%
\def\setversion{%
  {\versionfont{\the\version}}%
}%
\def\setdate{%
  {\versionfont{\the\date}}%
}%
\def\setlink#1{\red{#1}}%
%  
\def\setheading#1{%
  {\headingfont\blue{#1}}\raise\baselineskip\hbox{\special{pdf: outline 1 << /Title (#1) /Dest [
@thispage /FitH @ypos ]  >> }}}%
%
\def\setsubheading#1{%
  {\headingfont\blue{#1}}\raise\baselineskip\hbox{\special{pdf: outline 2 << /Title (#1) /Dest [
@thispage /FitH @ypos ]  >> }}}%
%
\def\link#1#2{\setbox0\hbox{\setlink{#1}}%
   \special{pdf: ann width \the\wd0\space height \the\ht0\space depth \the\dp0
     << /Type /Annot /Subtype /Link /Border [ 0 0 0 ] /A << /S /GoTo
     /D (#2) >> >>}\box0}
%
\def\dest#1{\special{pdf:dest (#1) [ @thispage /FitH @ypos ]}}%
%
\def\maketitle{\noindent\settitle\hfill\setauthor\par
          \hrule height 2pt\medskip
	  \noindent\setversion\hfill\setdate\vskip0.5in}
%
\def\section#1{\advance\sectioncount by 1\ssectioncount0
\bigskip\noindent\setheading{\the\sectioncount. #1}\par\nobreak\medskip}%
\def\subsection#1{\advance\ssectioncount by 1%
\bigskip\indent\setsubheading{\the\sectioncount.\the\ssectioncount\
	  #1}\par\nobreak\medskip}%
%
\def\display#1{\medskip\line{\quad #1\hfil}\medskip}
\def\example#1{\noindent{\it Example:\par\nobreak\noindent}}
%
\def\newpage{\vfill\eject}
%
\def\dvipdfm{{\tt dvipdfm}}%
\def\ttspecial{$\tt\backslash${\tt special}}%
%
\title{Dvipdfm User's Manual}
\author{Mark A. Wicks}
\version{Version 0.7dev}
\date{November, 1998}
\maketitle
\section{Background}
Currently, 
the most widely accepted method to generate PDF file from \TeX\
was to use Adobe's Acrobat distiller on a Postscript file produced by {\tt dvips}.
The hyperlink features are accessed by using \TeX\ \ttspecial s
to embed pdfmarks in the Postscript produced by {\tt dvips}.
H\`an Th\'e Th\`an's PDF\TeX\ project is an alternative solution.
Although quite good and fairly mature, the PDF\TeX project required
fairly significant modifications to
\TeX\ itself to add primitives to support the PDF features.
This project demonstrates that much of the functionality
of Acrobat Distiller can be achieved by using a DVI driver.
The PDF features are activated via \TeX\ \ttspecial s.

From a technical standpoint,
distiller will probably remain the
best approach for some time.
However, I have several objections to the use of distiller,
and feel that this driver provides a viable option.
One objection is that Acrobat Distiller isn't available for Linux---my principle
operating system.

My second objection is philosophical.
A DVI file is a page description.
It is essentially a linear program with no branching or decision instructions.
Postscript is a complete programming language, while PDF is a page description language
without any branching or decision capabilities.
\TeX\ is like postscript (without the graphics)
while DVI is like PDF (without the graphics or the hyperlinks).
Using Acrobat Distiller requires going from page description to program back to page description.
Pdfmarks are postscript features, which are meant for the distiller, are
analogous to \TeX \ttspecial s, which are meant for the DVI driver.
It seems natural to go directly from DVI to PDF, where \TeX\ replaces
postscript and where the DVI driver replaces and implements
\ttspecial s similar to the {\tt pdfmarks} in Adobe's Acrobat Distiller.

Unfortunately, until graphics software
begins to produce PDF content streams or encapsulated
PDF objects, Postscript will remain the easiest
way to include graphics in \TeX\ documents.
I would hope that in the future, graphics programs
will produce PDF content streams, or PDF objects that
may be included into a DVI to PDF translator.  Either
of these may be easily included using {\tt dvipdfm}
or a similar driver.

\section{General Concepts}
This document describes \dvipdfm\ driver.
It also servers as an example input file for
\dvipdfm, accessing some of the hypertext
features.  It assumes some familiarity with the basic features
of the \link{Portable
Document Format}{pdfmanual}.

Each \TeX\ \ttspecial\ 
constitutes a separate command to the \dvipdfm\ 
driver.  Each \ttspecial must begin with {\tt pdf:}
to identify that \ttspecial as a command for the \dvipdfm\ driver.
A \ttspecial\ beginning with any other characters is ignored
by the driver.  Leading spaces are ignored.  The characters {\tt pdf:}
are immediately followed by a \dvipdfm command.  These commands
are documented in Section~3.  Another feature of the driver
is variable expansion within PDF objects---specifically arrays and
dictionaries.  The driver maintains a symbol table.
Some of these variables are user defined and some are
driver defined (read only).  The read only
drivers are for referencing the current page, future pages,
or the current location on the page, for example.
User defined variables are references to user defined PDF objects.


\section{Dvipdfm Commands}
All commands are executed via \TeX\ $\tt \backslash${\tt special}s
prefixed with {\tt pdf:}, e.g.,

\display{\tt pdf: out 1 << /Title (Introduction) /Dest [ 1 0 R /FitH 234 ] >>
}

A complete list of commands follows:

\subsection{Ann}
The {\tt ann} command defines an annotation.  Annotations are typically
notes, hyperlinks, PDF forms, or PDF widgets.  The {\tt ann} command
takes the form:

\display{{\tt ann} [{\tt @}{\it name}] {\it dimensions}+ {\it dictionary}}

where name is an optional alphanumeric identifier and dictionary
is a valid PDF dictionary after variable expansion.
If {\tt @}{\it name} is specified, it may be used in
other PDF objects to refer to this annotation.
One or more {\it dimensions} parameters are required
and each consists of the keyword
{\tt height}, {\tt width}, or {\tt depth} followed
by an appropriate length, specified as per \TeX.  Each
length is a number followed by a unit, such as {\tt pt},
{\tt in}, or {\tt cm}.  A {\tt pt} is a \TeX pt, not a PDF pt.

\example
\display{\tt
  \ttspecial$\{$pdf: ann width 144pt height 10pt depth 2pt }%
\display{\tt \quad 
         << /Type /Annot /Subtype /Link /Border [1 0 0] }
\display{\tt \quad /A << /S /GoTo [ @nextpage /Fit ] >> $\}$ }


\subsection{out}
\display{{\tt out} {\it level} {\it dictionary}}

The parameter {\it level} is an integer representing the level of the outline
entry (beginning with 1) and {\it dictionary} must contain
the two keys {\tt /Title} and either {\tt /Dest} or {\tt /A}.
It may also contain the {\tt /AA} key.  These keys are documented
in the PDF Reference Manual.

\subsection{ARTICLE}

\subsection{DEST}

\subsection{docinfo}

{\tt docinfo} {\it dictionary}

The {\tt docinfo} command adds the keys in the specified dictionary to the
document's Info dictionary.  All keys are optional, but may include
the keys {\tt Author}, {\tt Title}, {\tt Keywords}, {\tt Subject},
and {\tt Creator}.

\subsection{docview}
{\tt docview} {\it dictionary}

The {\tt docview} command adds the keys in the specified dictionary to the
document's Catalog dictionary.  All keys are optional, but may include
the keys {\tt /PageMode},
{\tt /URI}, {\tt /OpenAction}, {\tt /AA}
and {\tt ViewerPreferences}.  See the PDF Reference Manual
for documentation of these keys and additional keys.

\subsection{epdf}

{\tt epdf} [@{\it name}] {\it filename}

The {\tt epdf} command ``encapsulates'' the first page of a PDF
file into a PDF XObject.  The resulting XObject is drawn
at the current location of the page.  The current point
represents the lower left-hand corner of the XObject's coordinate
system.  The optional @{\tt name} parameter may be used
to reference this object with other objects.
It will be expanded to a reference for this object
within any {\tt special} where a PDF object is expected.

\subsection{obj}

\display{{\tt obj} [@{\it name}] {\it object}}

The {\tt obj} command creates a
PDF object.  The parameter {\it object} is any valid PDF object.  The parameter @{\it name}
may be used to refer to this object within other objects.
It will be expanded within any {\tt special} where a PDF object is expected.
Typically {\it object} is an array or dictionary.  It may be an empty array or
dictionary that can be constructed dynamically via the {\tt put} command.

\subsection{put}
\display{{\tt put} @{\it name} {\it object}}

or 

{\tt put} @{\it name} {\it dictionary}

The {\tt put} command modifies an existing PDF object created with OBJ.
The first form is used when @{\tt name} is an array.  The second
form is used when @{\tt name} is a dictionary.  Arrays are
incremented one object at a time.  All keys in {\it dictionary}
are added to the dictionary represented by @{\it name}.

\subsection{close}

{\tt close} @{\it name}

The {\tt close} writes a PDF object created with OBJ to the PDF file.
No further PUT commands may be executed for this object.
The object may continue to be referenced using @{\it name}
indefinitely.

\section{Additional functions}
\subsection{bop}

{\tt bop} {\tt stream}

The {\tt bop} command specifies a marking stream to be generated at the top of each page.

\subsection{eop}

{\tt eop} {\tt stream}

The {\tt eop} specifies a marking stream to be generated at the top of each page.


\newpage
\section{References}
\dest{pdfmanual} Portable Document Format Reference Manual, Version
1.2, Adobe Systems Incorporated, 1996.  Available from {\tt
  http://www.adobe.com}. 


\bye
