% $Header: /home/mwicks/Projects/Gaspra-projects/cvs2darcs/Repository-for-sourceforge/dvipdfm-initial/doc/dvipdfm.tex,v 1.6 1998/11/19 15:28:35 mwicks Exp $
\special{  pdf:  docview <</PageMode /UseOutlines>>}%
\catcode`\ =11\def\space{ }\catcode`\ =10
% Page layout
\magnification\magstephalf
\advance\hoffset by 0.5truein
\advance\voffset by 0.5truein
\advance\hsize by -1.0in
\advance\vsize by -1.0in
%
%  Some helpful symbols
\def\rtm{{\font\r=cmss10 at 4pt
\font\c=cmsy5
\setbox0\hbox{\c\char13}\skip0\wd0\box0\setbox0\hbox{\r R}\advance\skip0 by
\wd0\kern-0.5\skip0\box0}}
\def\tm{{\font\r=cmss10 at 4pt \hbox{\r TM}}}%
%
%
% Some color definitions
% 
\def\colored#1#2{%
  \special{pdf:bc #1}#2\special{pdf:ec}}%
\def\red{[0.8 0.5 0]}%
\def\green{[0 1 0]}%
\def\blue{[0 0.4 0.8]}%
\def\yellow{[0.8 0.9 0.1]}%
\def\begincolor#1{\special{pdf:bc #1}}%
\def\endcolor{\special{pdf:ec}}%
%
% Some font definitions
%
\font\maintitlefont=cmti12 at 20.74pt
\font\headingfont=cmss12 at 14.4pt
\font\subheadingfont=cmss12 at 12pt
%
% Miscellaneous token lists
%
\newtoks\title\newtoks\author\newtoks\version
\newtoks\date
%
% Counters for section levels
%
\newcount\sectioncount\newcount\ssectioncount\newcount\sssectioncount
\sectioncount0\ssectioncount0\sssectioncount0
%
\newskip\indentlevel\indentlevel=\parindent
\def\beginlist{\par\advance\leftskip by \indentlevel\advance\rightskip by
\indentlevel\medskip}
\def\endlist{\par\advance\leftskip by -\indentlevel\advance\rightskip by
-\indentlevel\medskip}
%
%
\def\settitle{%
  {\maintitlefont\colored{\blue}{\the\title}}%
  \special {pdf:  docinfo << /Title (\expandafter\the\title) >>}}%
%
\def\setauthor{%
  {\headingfont\colored{\blue}{\the\author}}%
  \special {pdf:  docinfo << /Author (\the\author) >>}}%
%
\def\setversion{%
  {\subheadingfont{\the\version}}%
}%
\def\setdate{%
  {\subheadingfont{\the\date}}%
}%
\def\setlink#1{\colored{\red}{#1}}%
%  
\def\setheading#1{%
  {\headingfont\colored{\blue}{#1}}\raise\baselineskip\hbox{\special{pdf: outline 1 << /Title (#1) /Dest [
@thispage /FitH @ypos ]  >> }}}%
%
\def\setsubheading#1{%
  {\headingfont\colored{\blue}{#1}}\raise\baselineskip\hbox{\special{pdf: outline 2 << /Title (#1) /Dest [
@thispage /FitH @ypos ]  >> }}}%
%
\def\link#1#2{\setbox0\hbox{\setlink{#1}}%
   \special{pdf: ann width \the\wd0\space height \the\ht0\space depth \the\dp0
     << /Type /Annot /Subtype /Link /Border [ 0 0 0 ] /A << /S /GoTo
     /D (#2) >> >>}\box0}%
%
\def\dest#1{\special{pdf:dest (#1) [ @thispage /FitH @ypos ]}}%
%
\def\maketitle{\noindent\settitle\hfill\setauthor\par
          \begincolor{\red}\hrule height 1.0pt\endcolor\medskip
          \noindent\setversion\hfill\setdate\vskip0.3in}%
%
\def\section#1{\advance\sectioncount by 1\ssectioncount0
\goodbreak\vskip1.5\baselineskip\noindent\setheading{\the\sectioncount. #1}\par\nobreak\medskip}%
\def\subsection#1{\advance\ssectioncount by 1%
\bigskip\noindent\setsubheading{\the\sectioncount.\the\ssectioncount\
          #1}\par\nobreak\medskip}%
%
\def\display#1{\medskip\line{\quad #1\hfil}\medskip}
\def\example{\medskip\noindent{\it Example:\nobreak}}
\def\note{\medskip\noindent{\it Note:\par\nobreak}}
\def\syntax{\medskip\noindent{\it Syntax:\quad}}
\def\description{\medskip\noindent{\it Description:\quad}}
%
\def\newpage{\vfill\eject}
%
\def\dvipdfm{{\tt dvipdfm}}%
\input verbatim
\def\ttspecial{$\tt\backslash${\tt special}}%
%
\title{Dvipdfm User's Manual}
\author{Mark A. Wicks}
\version{Version 0.7dev}
\date{November, 1998}
\maketitle
\section{Introduction}
This package is a DVI (\TeX) to PDF conversion utility.
This package has the following features:

\beginlist
\item{$\bullet$} Support for outline (bookmark) entries, named destinations,
annotations (including forms and widgets).

\item{$\bullet$} Ability to include arbitrary PDF files as encapsulated
objects.

\item{$\bullet$} Ability to include JPEG images as encapsulated
objects.

\item{$\bullet$} A color stack.
\endlist
Currently, the widely accepted method to generate PDF file from \TeX\
is to use Adobe's Acrobat Distiller on a PostScript
file produced by {\tt dvips}.
The hyperlink features are accessed by using \TeX\ |special| primitives
to embed pdfmarks in the PostScript produced by |dvips|.
H\`an Th\'e Th\`an's PDF\TeX\ project is an alternative method
of generating PDF from \TeX source.
Although quite good and fairly mature, the PDF\TeX project
required modifying \TeX\ itself to add primitives that support the PDF features.
This |dvipdfm| project demonstrates that many of the features
of PDF can be accessed by using a DVI driver.
The PDF features are activated in the driver via \TeX\ \ttspecial primitives.

Even though Distiller is the best method of generating PDF (and
probably will remain so for some time) I have several reasons for
seeking alternatives to Distiller.
First, Distiller isn't available for my principle operating
system---Linux. Another objection is philosophical.
A DVI file is a page description.
Essentially, a DVI file is a program with no branching instructions.
PostScript is a complete programming language, while PDF is a page description language
without any branching or decision capabilities.
\TeX\ is like PostScript (without the graphics)
while DVI is like PDF (without the graphics or the hyperlinks).
Creating PDF from DVI using Distiller requires converting a page description to a program,
and converting that program back to a page description.
To continue this analogy,
Pdfmarks are PostScript ``escapes'' and are meant for the Distiller.
\TeX\ |special| primitives are \TeX\ ``escapes'' and are meant for the DVI driver.
It seems natural to go directly from DVI to PDF, where \TeX\ replaces
PostScript, where the DVI driver replaces Distiller,
and where \TeX\ |special| primitives replace the {\tt pdfmarks}.

Unfortunately, until graphics software
begins to produce PDF content streams or encapsulated
PDF objects, PostScript will remain the easiest
way to include graphics in \TeX\ documents.
I would hope that in the future, graphics programs
will produce PDF content streams, or PDF objects that
may be included into a DVI to PDF translator.  Either
of these may be easily included using {\tt dvipdfm}
or a similar driver.

\section{General Concepts and Syntax}
This document describes the |dvipdfm| driver.
The electronic version of the document exercises
some of the hypertext features and serves as
a sample input file for |dvipdfm|.
It assumes the reader has some familiarity with the basic features
of the \link{Portable
Document Format}{pdfmanual}.

Each \TeX\ |special|
represents a separate command to the \dvipdfm\ 
driver.  Each \ttspecial must begin with ``|pdf:|''
to identify that ||special|| as a command for the \dvipdfm\ driver.
A |special| beginning with any other characters is ignored
by the driver.  Leading spaces are ignored.  The characters ``|pdf:|''
are immediately followed by a \dvipdfm\ command.  These commands
are documented in Section~3.

\subsection{PDF Object Syntax and Variable Expansion}
With one exception, most of the syntax used within the specials follows
the PDF specification.
The single, necessary exception is variable expansion.
In the syntax specifications that follow, {\it PDF\_Object}
means that an arbitary PDF object is expected.  Similarly
{\it PDF\_Array} indicates that a PDF array is expected, {\it PDF\_Dict}
inciates that a PDF dictionary is expected, etc.
See the \link{reference manual}{pdfmanual}\ 
for a complete list of PDF object types.

The single extension implemented in this driver allows a symbol name of the
form |@|{\it name} whenever any PDF object is expected.
The {\it name} may contain the characters contained
in a PDF name and is delimited by white space.
A symbol beginning with |@| expands (if defined) to an indirect
reference to a PDF object.  This feature replaces
the |{|{\it name}|}| syntax used with pdfmarks.
Some of these named are user defined and some are
names defined by the driver.
The driver defined variables are for
referencing things like the current page, future pages,
or the current location on the current page.

The driver defined variables are
\par\medskip
\centerline{\vbox{\halign{{\tt #}\hskip
1em&\vtop{\leftskip0pt\rightskip0pt\hsize=3.0in\noindent #}\cr
\omit\hfil\it Variable\hfil&\omit\hfil \it Description \hfil\cr
\noalign{\smallskip\begincolor\red\hrule\endcolor\smallskip}
@thispage&An {\it indirect reference} to the current page.\cr 
@page&An {\it indirect reference} to page $n$.\cr
@nextpage&An {\it indirect reference} to the page following the current page.\cr
@prevpage&An {\it indirect reference} to the page preceding the current page.\cr
@ypos&A {\it number} representing the current vertical position in units of PDF points.\cr
@xpos&A {\it number} representing the current horizontal position in units of PDF points.\cr
}}}
\medskip

In the syntax specificatins that follow, several
standard conventions are followed.  Terminal
characters that appear in the command 
are typeset in the |tt| font, e.g., |object|.
Non terminal symbols are typeset in italics.
Optional parameters are surrounded by italic brackets, e.g.,
{\it [optional\_argument]}.  An item followed
by {*} represents an item that may appear
zero or more times.  An item followed by {\it +}
represents a required item that may appear multiple times.

\subsection{Dimensions and scalings}
Interaction with the |dvipdfm| driver consists
of short commands with few arguments delimited by white space.
Typically the arguments are PDF objects.
Two exceptions are dimension specifications and scalings.

In the \TeX\ style, a dimension specification consists of one of the keywords
|width|, |height|, or |depth| followed by a dimension
consisting of a numerical value, followed by a unit for the dimension.  The
unit will typically be |pt| (which represents a \TeX\ points, not a
PDf point) but |cm| and |in| are also allowed.  The notation
{\it dimension} in a syntax description means a dimension is expected.

A scaling consists of one of the keywords |scale|, |xscale|, or
|yscale| followed by a numerical value.  The notation
{\it scaling} means a scaling is expected. 


\section{Dvipdfm Commands}
All commands are executed via \TeX\ |\special| primitives
prefixed with the characters ``|pdf:|''.
\example
\begintt
\special{ pdf: out 1 << /Title (Introduction)
                        /Dest [ 1 0 R /FitH 234 ] >>
\endtt

\subsection{Annotate}
\syntax
{\tt annotate} [{\tt @}{\it name}] {\it dimension}+ {\it PDF\_dictionary}

\description
The |annotate| (|annot| or |ann|) command defines an annotation.  Annotations are typically
notes, hyperlinks, PDF forms, or PDF widgets.
The parameter {\it name} is an optional alphanumeric identifier
and {\it PDF\_dictionary} is a valid PDF dictionary after variable expansion.
If {\tt @}{\it name} is specified, it may be used in
other PDF objects to refer to this annotation.
One or more {\it dimension} parameters are required
and each consists of the keyword
{\tt height}, {\tt width}, or {\tt depth} followed
by an appropriate length, specified as per \TeX.  Each
length is a number followed by a unit, such as {\tt pt},
{\tt in}, or {\tt cm}.  A {\tt pt} is a \TeX\ pt, not a PDF pt.

\example
\begintt
\special{pdf: ann width 144pt height 10pt depth 2pt 
           << /Type /Annot /Subtype /Link /Border [1 0 0] 
              /A << /S /GoTo [ @nextpage /Fit ] >> }
\endtt


\subsection{Article}
\syntax
{\tt article} {\tt @}{\it name} {\it PDF\_dictionary}

\description
The |article| (or |art|) command initializes an article.  An article
is a collection of boxed regions in the document that should be
read consecutively. The {\it name} parameter is required.  The required PDF dictionary 
is similar to the |docinfo| dictionary and
should include the |/Title| and |/Author| keys.
\example
\begintt
\special {pdf: article @somearticle << /Title (Some title)
                                       /Author (Me) >>}
\endtt

\subsection{Bead}
\syntax
{\tt bead} {\tt @}{\it name} {\it dimension}+

\description
The |bead| command adds a rectangular area to an existing article thread.
The parameter {\it dimension}+ specifies a rectangular area
in the same manner as for an annotation.  The {\it name}
must correspond to an existing |article|.
\example
\begintt
\special{pdf: bead @someart width 156pt height 20pt depth 4pt}
\endtt

\subsection{Dest}
\syntax
{\tt dest} {\it PDF\_String} {\it PDF\_Dest}
\description
The |dest| command defines a named destination.
The {\it PDF\_String} is a PDF string naming
the destination.  This string may be used in the destination
fields of annotations and outline entries to refer to
this destination.  {\it PDF\_Dest} is a PDF
destination object (typically an array).
\example
\begintt
\special{pdf: dest (listofreferences) [ @thispage /FitH @ypos ]}
\endtt

\subsection{Docinfo}
\syntax
{\tt docinfo} {\it PDF\_dictionary}
\description
The |docinfo| command adds the keys in the specified dictionary to the
document's Info dictionary.  All keys are optional, but may include
the keys |/Author|, |/Title|, |Keywords|, |Subject|,
and |Creator|.
\example
\begintt
\special{pdf: docinfo << /Author (Mark A. Wicks)
                         /Title (This Document) >>}
\endtt

\subsection{Docview}
\syntax
{\tt docview} {\it PDF\_dictionary}
\description
The |docview| command adds the keys in the specified dictionary to the
document's catalog dictionary.  All keys are optional, but may include
the keys |/PageMode|,
|/URI|, |/OpenAction|, |/AA|
and |/ViewerPreferences|.  See the PDF Reference Manual
for documentation of these keys and additional keys.
\example
\begintt
\special{pdf: docview << /PageMode /UseThumbs >> }
\endtt


\subsection{Object}
\syntax
{\tt object} [@{\it name}] {\it PDF\_Object}
\description
The |object| (also |obj|) command creates a
PDF object.  The parameter {\it PDF\_Object} is any valid PDF object.  The
parameter {\it name} may be used to provide an indirect reference
to this object within other
objects. It will be expanded anywhere within
a {\tt special} where a PDF object is
expected. Typically {\it object} is an array
or dictionary.  It may be an empty array or
dictionary that can be constructed dynamically via
the {\tt put} command.
\example
\begintt
\special{pdf: object @mydict << /Firstpage @thispage >>}
\endtt

\subsection{Out}
\syntax
{\tt out} {\it number} {\it PDF\_dictionary}

\description
The |out| (also |outline|) command adds an outline (also called a ``bookmark'') entry
to the document.
The parameter {\it level} is an integer representing the
level of the outline entry (beginning with 1) and
{\it  PDF\_dictionary} must contain
the two keys {\tt /Title} and either {\tt /Dest} or {\tt /A}.
It may also contain the {\tt /AA} key.  These keys are documented
in the PDF Reference Manual.
\example
\begintt
out 1 << /Title (Section 1) /Dest [ @thispage /FitH @ypos ] >>
\endtt
which may be followed by
\begintt
out 2 << /Title (Section 1.1) /Dest [ @thispage /FitH @ypos ] >>
\endtt
\note
You may not skip levels.  A level~2 outline entry
must follow a level~1 outline
entry.  A level~3 outline entry must follow a level~2 outline
and cannot immediately follow a level 1 outline entry.

\subsection{Put}
\beginlist
{\tt put} @{\it name} {\it PDF\_Object}
\endlist
or 
\beginlist
{\tt put} @{\it name} {\it PDF\_Dictionary}
\endlist
\description
The |put| command modifies an existing PDF object created with |obj|.
The first form is used when @{\tt name} is an array.  The second
form is used when @{\tt name} is a dictionary.  Arrays are
incremented one object at a time.  All keys in {\it PDF\_Dictionary}
are added to the dictionary represented by @{\it name}.
\example
\begintt
\special{pdf: object @mydict << /Nextpage @thispage >>}
\endtt

\subsection{Close}
\syntax
{\tt close} @{\it name}
\description
The |close| writes the named PDF object created with |obj| to the PDF file.
No further |put| commands may be executed for this object.
The object may continue to be referenced using @{\it name}
indefinitely.  If the object is never closed, it will
be closed when |dvipdfm| finishes processing the document.

\section{Color Commands}
\subsection{Begincolor}
\syntax
{\tt begincolor} {\it PDF\_Array}
\description
The |begincolor| (|bcolor| or |bc|) command uses the
array to set the default color for future marking operators.
The current color is pushed on the color stack.  The
array must have three elements specifying the coordinates
of the color in the Device RGB color space.
\example
\begintt
\special{ pdf: bc [ 1 0 0 ] }
\endtt

\subsection{Endcolor}
\syntax
{\tt endcolor}
\description
The |endcolor| (|ecolor| or |ec|)
changes the default color to
match the color on the top
of the stack.  It removes
the color from the stack.
\example
\begintt
\special{ pdf: ec }
\endtt


\section{Image Commands}
\subsection{Epdf}
\syntax
{\tt epdf} [|@|{\it name}] [{\it dimension}$\vert${\it scaling}]*  {\it PDF\_String}

\description
The {\tt epdf} command ``encapsulates'' the first page of a PDF
file named by {\it PDF\_String}
into a PDF XObject.  The resulting XObject is drawn
with the lower left corner at the current location of the page.
The optional @{\it name} parameter may be used
to reference this object within other objects.  If a
{\it dimension} is supplied, the object will be scaled to fit
that dimension.  A {\it scaling} consists of one of the keywords
|scale|, |xscale|, or |yscale| followed by a number representing
the scaling factor.  Both {\it scaling} and {\it dimension}
parameters can be supplied as long as they are not logically
inconsistent.
\example
\begintt
\special{pdf:epdf yscale 0.50 width 4.0in (circuit.pdf)}
\endtt

\subsection{Image}
\syntax
{\tt image} [@{\it name}] [{\it dimension} $\vert$ {\it scaling}]*  {\it PDF\_String}

\description
The {\tt image} command ``encapsulates'' a JPEG image
taken from the file named by {\it PDF\_String}.
Otherwise, this command functions just like |epdf|.

\section{Raw Page Marking Commands}

\subsection{Content}
\syntax
{\tt content} {\tt stream}
\description
The |content| command specifies a marking
stream to be added to the current page at
the current location.  While it
is possible to change the color
state, etc., with this command, it is
not advised.  Use the color management
commands to change colors.

\subsection{Bop}
\syntax
{\tt bop} {\tt stream}
\description
The |bop| command specifies a marking
stream to be generated at the top of each page.
The parameter {\it stream} is any sequence
of marking operators and is added to the page's content stream.  
The stream is applied {\it to all pages} regardless
of where it appears in the document.
\example
\begintt
\special {pdf: bop  q 0.8 0.5 0 RG 0 0 m 612 0 l
                                   612 792 l 0 792 l b Q }
\endtt
\special {pdf: bop  q 0 w 0.8 0.5 0 RG
                           54 740 m 504 740 l 504 740.25 l 54 740.25 l b  
                           36 760 m 504 760 l 504 760.25 l 36 760.25 l b Q }


\subsection{Eop}
\beginlist
{\tt eop} {\tt stream}
\endlist
The {\tt eop} specifies a marking stream to be generated at the end
of each page. The parameter {\it stream} is any sequence
of marking operators and is added to the page's content stream.  
The stream is applied {\it to all pages} regardless
of where it appears in the document.

\section{Examples}
The following image was included from a JPEG file:

\centerline{\vbox to 1.42in{\hrule\vfil\hbox to 4.37in{\special{pdf: image
(gnome.jpg)}\hfil}}}

The following image is identical, but loaded 
with |scale 0.25|.

\centerline{\vbox to 0.36in{\vfil\hbox to 1.10in{\special{pdf:
image scale 0.25
(gnome.jpg)}\hfil}}}

Graphics work, but you need to put the graphic image
in your own box of the correct size so \TeX\ knows
about it.  No space is reserved for a |special|
unless you reserve it.



\newpage
\section{References}
\item{[1]}\dest{pdfmanual}Portable Document Format Reference Manual, Version
1.2, Adobe Systems Incorporated, 1996.  Available from {\tt
  http://www.adobe.com}. 

\bye
